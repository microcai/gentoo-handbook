

\section{搭建 HTTP 服务器}

首先要做的是先安装一个HTTP服务器。Gentoo在portage里收录了很多HTTP服务器。这里自然说的是LAMP\footnote{Linux、Apache 、 MySQL、 PHP}里的Apache HTTP Server。

\begin{example}{安装Apache}
\begin{code}
\# emerge apache -av
\end{code}

由于apache高度模块化,各个功能被拆分到了不同的模块里。而Gentoo独有的USE机制就允许用户只编译需要的模块。
安装 apache 会出现类似下面的提示
\begin{code}
[ebuild  N     ] app-admin/apache-tools-2.4.3-r1  USE="ssl" 4,453 kB

[ebuild  N     ] www-servers/apache-2.4.3:2  USE="ssl -debug -doc -ldap (-selinux) -static -suexec -threads" APACHE2\_MODULES="actions alias auth\_basic authn\_alias authn\_anon authn\_core authn\_dbm authn\_file authz\_core authz\_dbm authz\_groupfile authz\_host authz\_owner authz\_user autoindex cache cgi cgid dav dav\_fs dav\_lock deflate dir env expires ext\_filter file\_cache filter headers include info log\_config logio mime mime\_magic negotiation rewrite setenvif socache\_shmcb speling status unique\_id unixd userdir usertrack vhost\_alias -access\_compat -asis -auth\_digest -authn\_dbd -cache\_disk -cern\_meta -charset\_lite -dbd -dumpio -ident -imagemap -lbmethod\_bybusyness -lbmethod\_byrequests -lbmethod\_bytraffic -lbmethod\_heartbeat -log\_forensic -proxy -proxy\_ajp -proxy\_balancer -proxy\_connect -proxy\_ftp -proxy\_http -proxy\_scgi -reqtimeout -slotmem\_shm -substitute -version" APACHE2\_MPMS="-event -itk -peruser -prefork -worker" 24 kB
\end{code}

可以在make.conf里设定APACHE2\_MPMS和APACHE2\_MODULES的值。初次配置apache的同学可能要犯傻了,这么多模块! 我咋知道我需要哪个呢?

拿不准的情况下就全编译了吧(笑)。

\end{example}




\subsection{  apache 用的最多的服务器}
\subsection{ nginx 轻量级服务器}
\subsection{ lighttpd 超轻量级服务器}
\subsection{ squid 加速代理	}
\section{  数据库	}
\subsection{  最流行的开源数据库 MySQL}
\subsection{  最优秀的开源数据库 PostgreSQL 	}
\subsection{  商业霸主 OracleDB	}
\section{  加速 DNS ,在本机搭建 DNS	}
\section{  共享打印机	}
\subsection{ CUPS 打印服务	}
\subsection{ Samba 打印机共享	}