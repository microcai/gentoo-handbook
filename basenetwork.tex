
%\begin{quotes}[0.5]{microcai}
%The protocol is the kernel of the net. The kernel is the protocol of the UNIX. The UNIX is the source of the net.The source is the kernnel of the UNIX.UNIX, the net , the protocol , you are one.
%\end{quotes}

若要问世纪末最伟大的发明是什么,一定有人说,是Internet。Internet让全世界的电脑连在了一起。

\section{层次协议网络}

说到网络不得不提到OSI的7层网络模型。

\begin{table}[h]
\begin{center}
\caption{OSI模型}\label{tabel:ISOmodule}
\begin{tabular}{|c|c|c|c|}
\hline
& 数据单元 & 层 & 功能  \\ \hline % \cline{2-4}
\multirow{4}{*}{主机层}   &  \multirow{3}{*}{块数据} & 应用层  &  网络应用程序  \\ \cline{3-4}
&						 & 表现层 & 数据表现,加密解密,数据格式转化  \\ \cline{3-4}
&						 & 会话层 & 管理会话 \\ \cline{2-4}
& 分片数据  & 传输层 & 点对点的可靠通讯,流控制 \\ \hline
\multirow{3}{*}{媒介层}  & 数据包   & 网络层 & 网络寻址和路由 \\ \cline{2-4}
& 帧 & 链路层 & 物理寻址和交换 \\ \cline{2-4}
& 比特 & 物理层 & 物理信号传递:光电等 \\\hline
\end{tabular}
\end{center}
\end{table}

OSI定义的7层模型是参考模型\footnote{实际上构成互联网基石的TCP/IP协议并没有完全按照7层实现,而是只有4层。}。

第一层,物理传输层。典型的我们每天都能见到的就是以太网和Wi-Fi了。太网使用同轴双绞线传递差分信号,由以太网协议规范。网卡的接口电路负责实现。
使用同轴双绞线传递差分信号,可以认为能传递 0 和1。Wi-Fi则使无线信号进行传输。物理传输层的单位是比特。我们常说的100M网卡指的就是网卡使用的100M以太网协议,每秒可以发送100M个比特的数据。这个数据是指物理传输层的。56k调制解调器也是说的这一层的速率为56Kbit/s。

%操作系统无法管理到。不属于TCP/IP管辖范围。

第二层,链路层。典型的链路层协议就是以太网了\footnote{什么?又是以太网? 额,不是说了么,实际上没有完全参照OSI的模型。}。以太网协议包含了物理传输层和链路层两层协议。
在链路层,以太网使用“以太网帧”封装形式向上层提供基于帧的数据包传输功能。传输的单位是一个以太网帧。 以太网帧具体长度不确定,以太网协议只规定了最小和最大数据包长度。典型网络应用下,以太网最大帧长度为1518字节,最小帧长度为64字节。除去以太网帧的头部信息,以太网帧最大可以传输的有效数据为1500字节。

第三层,网络层。当下用的最广泛的网络层协议就是IP协议和它的升级版本IPv6协议\footnote{不过在不混淆的情况下,可以继续IP协议指的是IPv4。}。
IP协议发送的依然是数据包。每个数据包由“IP协议头+负载数据”构成。一个IP数据包正好作为链路层(比如以太网)的负载数据发送。IP协议为每一台主机分配了一个全球唯一的IP地址\footnote{实际上IP地址划分出了几个区段,不能用于因特网,只能用于自己组件的私有网络中。这类私有地址在全球范围内是可以重复的。但是在每个私有网络内部仍然是不能重复的。}。IP地址通常使用点分十进制表示法。

第四层,传输层。自然就是TCP协议了。TCP在IP包的基础上提供了可靠的流式传输协议。TCP将数据打散为一个一个的包,然后利用IP协议传输打散的包。
如果发生丢包,TCP协议还会重新传输丢弃的包。到目的地,TCP会将数据包重新组合为数据流。另一个传输层协议是UDP协议。 %不过只简单的把IP协议包装一下,所以不能提供可靠

第五层,会话层。还是TCP协议。TCP包含了连接建立,数据收发和连接断开。建立一个连接就是一次TCP会话。TCP协议使用“三次握手”建立连接。断开连接则是四次握手。

第六层,表现层。在TCP/IP中,这个层已经不是他们负责的啦。如果应用程序使用了SSL加密协议,比如使用 HTTPS 访问网站,则SSL可以认为是这一层的。
直接传递数据而不加密的应用中,这一层是缺失的。

第七层,应用层。这类协议众多。访问网站使用的HTTP协议,访问FTP使用的是FTP协议,等等。

在每一层协议中,都有一个“负载”用来承担上一层协议的数据。理论上来说每层都是独立的,没有一定要用第$n$层来装配第$n+1$层数据的说法。比如链路层可以不建立在物理传输层之上,而是依靠其他层的协议建立一个“虚拟”链路。典型的应用就是VPN(虚拟专用网)。
%而实际上,层之间甚至是可以交叉越级的。比如各种VPN\footnote{后面会介绍VPN是什么}协议就是利用的应用层来传输IP数据包。还有利用应用层传输以太网帧的。

\section{物理传输层和链路层}

这两层所以放到一起,是因为在IP网络中,这两层被看作为一层:网络接口层。向上层IP协议提供分组交换的网络。
分组交换的意思是,通信双方的传输单位是分组的数据包,每个数据包要么完整传达,要么整体丢弃。数据包在存在“交换设备”的情况下可以在交换设备中依据地址重新转发。

\subsection{以太网}

说起网络接口,大家都对RJ45接口不陌生。这个插头的符号都要成为网络接入的代名词了。因为它正是被大名鼎鼎的“以太网”选用的标准插接口。

\chatu[width=0.7\textwidth]{Ethernet_RJ45_connector_p1160054}{RJ45网线插头\protect\footnotemark}

\footnotetext{图片来自维基百科}

以太网,英文名Ethernet,是使用的最广泛的网络。从最初的10M/s传输速度到100M/s再到1000Mb/s,甚至10G/s的传输速率都研究出来了。现在应用最广泛的还是100M/s和1000M/s两种。
以太网最初由施乐公司开发,于1985年成为IEEE 802.3标准。此后横扫全球,将所有其他的有线网络一网打尽。

具体的来说,以太网定义了两个子协议,分别定义在物理和软件两个层次。物理层次来说,以太网规定了使用双绞线以差分电信号的形式传递基带信号\footnote{通信业的行话,未经过调制的数字信号就是基带信号。简单的理解就是0和1嘛。}。

在这个层,以太网发送的数据以以太网帧的形式进行。最大一次可以发送 1500Byte的有效数据。在链路层,以太网协议负责实现数据包的传递。每个数据包由“前导同步序列+太网帧头+负载数据+尾同步序列”构成。
同步序列用来在01比特流中识别出以太网帧。并且网卡在发送同步序列的时候还负责检测“碰撞”——多个网卡同时向一条线路上发送以太网帧。
遇到碰撞的时候,网卡负责重发。以太网帧头包含了负责数据的大小、发送方地址和接受方地址。
每个以太网接入设备\footnote{接口卡的话被称为网卡}被分配了全球唯一的地址。这个地址被称为MAC地址。操作系统能控制的就是以太网头部中的“发送地址和目的地址”以及负载数据。
这一层依旧使用以太网协议。

\subsection*{小插曲:永恒的以太网}
以太网之父层说过,如果有什么技术要取代以太网,那么取代以太网的技术也会被称为以太网,所以以太网是永远不会消失的。最初的以太网标准是 10BASE5, 传输速度只有10M/s。但是100M/s我有线网络开发出来后,本来取代以太网的协议依旧称呼为以太网——快速以太网是也。
1000M/s的网络开发出来了,本来要取代以太网的,结果又冠名为千兆以太网。果然被以太网之父说中了。以太网是永恒的。

以太网不仅征服了有线网络,还有无线网络。Wi-Fi,又被称呼为wireless-lan,无线局域网,另一个名字就是无线以太网。什么?又是以太网?是的,没错,依旧是以太网。
以太网征服了有线网络和无线网络,还不肯罢休,因为它又征服了光纤网络。EPON\footnote{Ethernet Passive Optical Network,以太无源光网络。}又被称呼为最后一英里的光纤。从前从运营商到用户都是使用的ASDL这里设备,EPON让光纤跑起了以太网协议,代替ASDL\footnote{所谓光纤入户就是EPON。}。%,使得PPPoE\footnote{PPP Over Ethernet。}协议可以在光纤上直接运行。
是的,EPON使用光纤作为传输介质,但是还是以太网协议。

还有 10G/s 速率的以太网,更是已经将光纤接口纳入标准。是的,还是以太网,不过传输介质已经是光纤了。

以太网所以能广泛应用,得益于它的简单和有效。相比于其他协议\footnote{作为对比,可以先阅读一下ATM相关资料。},以太网协议非常简单。以太网协议将自己划分为链路层和物理传输层两个子协议。链路层部分以太网几十年如一日的保持了简单和稳定。不管是有线还是无线甚至是光纤,链路层的协议永远都是一样的。唯一变化的是传输层。从10M/s到10G/s,变化的永远是传输层。
传输层规定了具体的0和1如何传递。它可以在五类线上定义一种电信号模式,在双绞线上使用另外的电信号模式,在光纤上又能定义新的光信号模式,但是其上的链路层永远是一致的。
永恒的是以太网,永恒的是链路层以太网协议。

\subsection{无线局域网Wi-Fi}

\subsection{点对点协议}

另外一个重要的链路层协议是PPP。PPP是个链路层协议,也就是第二层,那么谁给PPP提供第一层呢?如果第一层是电话线,那么PPP就是构建了拨号网络,如果第一层是以太网,那就是大名鼎鼎的PPPoE(PPP over Ethernet)。

\subsection{虚拟局域网-VPN}

看上去是个应用程序的虚拟网其实是链路层的。

\section{网络层:IP协议}

IP协议从OSI的模型上看,属于网络层。使用IP协议的网络就是IP网络\footnote{除了IP网外,还有电话网,有线电视网。}。
IP网络按照规模分为局域网,城域网和广域网。而当今世界上最大的广域网就是因特网。

IP网的作用类似快递——传递数据包。IP网并不保证数据包总是能到达目的地——虽然网络建设的时候总是力求降低丢包率,但是IP本身并不100\%保证数据传达。

IP网工作在网络层,需要使用链路层协议传输数据包。在全球范围内,IP骨干网使用的是ATM和X.25这样的光纤网络进行传输的。在本地局域网,使用以太网协议。从ISP到家庭用户,使用的是
ADSL这样的拨号网络。甚至可以不是物理网络:使用各种应用层协议传输IP包——这就是VPN的原理了。

由于IP协议不提供可靠的数据传输。IP只能尽量保证将数据包传递到目的地。在这里不可靠的意思有两层:第一个意思是数据包可能丢失,第二个意思是丢失不丢失发送方无从知晓。如果发送方知道数据丢失,则仍然任务协议是可靠的——发送方可以选择重发或者标记网络不可达——这意味着任何协议都无法完整数据包的送达。

IP的另一个不可靠特性是接受方收到的数据包次序和发送方并不能保证是一致的。可能会出现乱序。因为IP数据包是经过“路由”的。我们会在下一个章节介绍路由。

\subsection{路由表和路由协议}

上节说到IP协议传递的是分组的数据包,每个数据包标记了目的地址和发送地址。数据包在发送到目标之前会经过层层的“路由”。

TODO 路由示意图

解说: IP网络由主机和路由器以及连接它们的通讯线路组成。图中示意的是一台主机发送数据包到多台目标主机的过程。

主机A分别发送数据包到B、C、D,向B是直接发送。向C是通过B的转发到达。而发送到D的数据则经过了E、F、H的转发。另外一条可能的路径是B->G->F->H->D,不过这是一条比较“远”的路径。
像 F 这样的主机连接了大量的网络,专门用来转发数据包,被称之为路由器。B这样的主机,本身也提供服务,但是也转发数据包。在转发数据包的时候它是路由器,本身发送和接收数据包的时候它是主机。

一台主机是不是路由器主要看它在网络上的作用。我们有时候会把专门制造的用于路由的计算机称作路由器。当是IP网络上任意一台主机都可以担当路由器——只要它扮演了这个角色。

我们可以看到,一个数据包从A到D可以选的路径不止一条——但是有的路径经过的节点少,有的则多。也就是说,这些路径的质量是不能等同看待的。
衡量一个路径的标准自然就是用户最关心的:延迟和带宽。经过的节点越少通常来说延迟就越少。带宽自然是越高越好——为了尽量避免拥挤的线路,不仅仅是要选择带宽高的路径,还要选择带宽没被占满的。

如此一来就带来两个问题:第一个问题,IP数据包在进行路由的时候,如何知道哪条路径是可以将数据包准确的送达目的地的——这属于路由发现问题;第二条,拥有多个路径可选的时候 ,如何选择一个最佳路径——这属于路由选择问题。

在操作系统的实现上,内核是依据一个“路由表”执行路由的。路由表可以修改,所以选择最佳路径归结为“在系统运行中动态的修改路由表以让内核总是使用最佳路径”这个问题了。

路由表是一张表格,详细的列举了每个IP(也可以是一个区间,这样可以减小表的大小)应该使用哪个路径。比如对A来说,向C和B发送的数据,使用的网关就是B。路由表上记载的应该就是 C->B , B->B 这样对应的条目。但是对于D,路由表包含的是 D->E 。使用 E 来转发。而事实上E并不能直接将数据转发给D,E 的路由表里包含 D->F 这样的条目,因此E将数据包交给 F 来转发。数据就是这样层层转发的,路由表并不记录完整路径,而是接力棒一样。
所以用来接下一棒的路由器被形象的称呼为网关。

路由表可以由管理员手工输入(使用route(8)命令)。管理员输入的路由表被称为静态路由表。但是面对复杂的网络——尤其是那些运行在互联网的骨干上的路由器来说——手工维护显得非常不合理。
所以就催生了各种路由协议,在路由器之间动态维护路由表。粗略的列一下,大概有边界网关协议(BGP)、外部网关协议(EGP)和内部网关协议(IGP)。
大协议下还细分了许多子协议,用来进行距离探测,链路状态探测等。路由协议运行在互联网的骨干路由器上。用于路由器之间维护完整的优化的路由表。普通用户的机器通常并不需要运行路由协议。设置一个默认网关就足够了。

\begin{insertnote}
\subsubsection{默认网关}

默认网关告诉操作系统,所有条目都不匹配的时候使用这个网关。因为路由表总是从上而下进行匹配的,所以默认网关总是路由表的最后一条。

一般家庭局域网的机器上,路由表只有两条:一条告诉本机哪个IP范围的IP是直接发送的,其他的都使用默认网关。属于本地局域网的IP自然是不经过路由的。机器判定本地IP的范围使用的就是
DHCP\footnote{参见\secref{sec:DHCP}}的时候获得的掩码。而大部分IP地址都是用过默认网关进行路由的。毫无疑问这个默认网关就是家庭路由器:)。

\end{insertnote}

那么,接下来整理一下,我们在这个想象中的网络具体的观察一下IP协议的运转。

TODO 插图
%TODO

\begin{enumerate}
\item [局域网(Hub集线器)] \item [局域网(带交换机)] 主机A发送到主机B,因为是他们连接到一个集线器上,所以主机A首先发送一个以太网广播(ARP协议)获得B的mac地址。
然后就可以直接向B发送数据了。Hub集线器实际上就是把网线加终端电阻连接到一起。所以数据在Hub的端口之间是直接广播的。目的地址是B的mac地址的以太网帧被B收到——也就是收到A发送的IP包了。

\item [局域网(带交换机)] 主机A发送到主机C,因为是他们连接到同一台交换机上,所以主机A首先发送一个以太网广播(ARP协议)获得C的mac地址。
然后就可以直接向C发送数据了。数据由交换机交换传递到C连接到的端口,这样主机C的网卡就能收到A发送的IP包了。

\item [局域网(无线网桥\footnote{简称AP。})] 主机A发送数据包到D,中间通过了无线网桥,将有线以太网协议转化为802.11a/b/g协议,使用天线发送。802.11a/b/g协议属于第一层协议。第二层依然是以太网协议。网桥在这里是做了
第一层有线以太网协议到802.11a/b/g协议的转化。IP是第三层协议,所以这个时候依赖的第二层以太网协议本身没有变化。对IP层来说,无线和有线是透明的。都是以太网。所以这里依然使用ARP协议获得接收方的网卡地址,然后直接发送。这里网桥和交换机是一样的功能。

\item [局域网(无线到无线)] 主机A发送数据到D,这个时候断开网线,自动切换到使用无线网络。发送依然要经过网桥。仍然是以太网,依旧是ARP协议首先获得对方的mac地址,然后发送。数据由网桥转发到目的主机。

\end{enumerate}

\section{传输层:TCP和UDP}
为了提供可靠的通信,可以在IP协议的基础上加入数据接受确认、超时重发、和顺序重组功能就可以提供一个可靠的面向流的协议。该协议提供给上层可靠的面向流的协议——可靠的意思是,协议告诉你发送成功接收方就一定已经收到数据,协议告诉你发送失败接收方就是没接收到;面向流的意思是,数据在底层会被分包以使用IP协议传输,但是如何分包是不需要上层操心的,上层交给协议的 就是一连续的字节流,接收方收到后会重组为正确次序的字节流。除了可靠和流式这两个特点,还应该加入拥塞控制——发送速度不要超过网络能提供的最大带宽避免网络拥塞。整合这些需求后,诞生的就是TCP协议了。

由于主机通信双方并不可能只有一个程序。所以当一个TCP数据送达,只根据发送方地址来区分(这样的话一个主机只能向另一个发起一个链接了)太不合理了,所以TCP协议引入了“端口”的概念。
如此一来,一个主机就可以执行多个程序,使用不同的端口进行通信,相互之间不干扰。这样确定一条TCP链接的就是两个IP地址和2个端口了\footnote{发送方和接收方的端口,发送方和接收方的IP。}。如果没有端口,则两台主机之间只能建立一条TCP连接。

如果应用程序不需要TCP提供的可靠流式传输,想直接使用IP这样简单的“尽力送达”服务,另外有一个UDP协议可以使用。UDP协议在IP协议之上简单的加了一层“端口”。当然加入端口和TCP的原因是一样的——没有端口的话,一台主机岂不是只能有一个程序能接收数据?\footnote{操作系统将无法确定接收到的数据包应该传递给哪个程序。
有了端口,使用UDP协议的程序可以绑定到一个端口上,这样对应的端口接收到数据操作系统就可以知道将数据递交给哪个程序。}

%\section{虚拟局域网}