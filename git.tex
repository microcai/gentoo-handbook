
版本控制是什么?我们为什么需要它?
只要用电脑进行文件编辑,就不能避免的要处理修订历史。很多人选择拷贝多个备份,这是一种非常原始的版本控制。一旦工作开始变得烦琐,人就乐意开发工具来自动化的进行这种版本修订的历史记录工作。
其中最出名的工具,恐怕就是Git了,它为什么这么流行?它能做什么?到底如何使用?本章将揭示这些。

%一个人的时候,这种原始的版本控制顶多会让你决定很麻烦,很混乱。
%但是如果两个人以上的团队共同编辑一个项目,版本控制软件就是必须的了。多数版本控制是用来管理软件开发的,但是版本控制用于普通的文件编辑也可以的,像是美术设计啊,小说创作啊,等待。你看到的本书就是使用版本控制进行管理的。
%,只不过在程序员的圈子外少有人用版本控制。\footnote{可惜的是他们享受不到版本控制的好处。}

\section{起步}
\subsection{历史和后悔药}

从手指接触键盘伊始,手中被编辑的文件就有了版本。随着时间推移,很多东西加了进来,也有很多东西被Del请了出去。你一定会对你用着的靠谱的编辑器点下那很靠谱的“\includegraphics[width=1em]{pics/document-save.png}”按钮。 % 好的,有一天,突然想起来一个月前删除的那段代码/文字/数据... 需要恢复。
关闭编辑器后,你发现你写错了,需要修改。或者说,你迫切的想回退到原来的那个版本。
怎么办?一般的编辑器有撤销按钮,可是关闭了编辑后再打开,可从来就没有编辑器还能再撤销的。
%这个时候的版本控制就是你的后悔药。就算你把整个项目的文件都删除,版本控制都能轻轻松松的帮你找回所有的文件。

幸运的是,你可能一早就想到了解决办法——备份。因为你有一个非常好的习惯:在编辑任何文件前备份。

\begin{code}
\# cp -R myproject myproject.bak
\end{code}

很快你会发现,保留多个备份会很大程度上方便你的编辑。命令变成了以日期为后缀的备份

\begin{code}
\# cp -R myproject myproject-2012-12-21.bak
\end{code}

用不了多久,你的电脑里就到处都是各种日期的备份文件了。这些备份文件,不仅仅构成了你的后悔药,更重要的是,它也是历史。

我想你很快就会觉得那么多历史开始变得讨厌起来了,它不仅仅占用了大量的磁盘空间,还干扰视线。特别是当你寻找什么东西的时候。

你很快就想删除历史,只保留最近的几份了。别着急,马上会有解决办法,在次之前,先看另一个场景,可能并不是所有人都需要,不过姑且看看。
%像内核那么庞大的项目, 很难想像如果没有版本控制会是什么情况。有了版本控制,文件的每一行修改都可以追溯。如果发生bug,也少了嘴皮之争,版本控制明明白白的写清楚了,谁在什么时候修改了哪个文件,修改了哪些行。

\subsection{diff 和 patch}

很快你的好朋友对你做的东西产生了兴趣,于是你把项目打包了一份给了他。不久,他要么发现了你的几个错误,要么自己添加了一些东西,总之他对你的东西进行了修改。你发现他的修改非常有用,于是要合并他的修改。显然,让你的朋友一个一个的告诉你他修改了哪里并不太容易,于是你简单的要他把修改后的项目打个包给你。

接着你需要知道他修改了哪里。也就是说,你需要比较一下他修改后的文件和你发给他的那份之间的区别。这个时候就是祭出大名鼎鼎的 diff 的时候了。

\begin{code}
\# diff -uR modifiedproject  myproject-2012-12-21.bak
\end{code}

diff 的输出非常容易理解, +打头的表示添加了一行,-打头的表示删除了一行。删减的前后还会引用数行修改地点的上下文。这样就能很容易的理解修改的地方了。

下面是本文写作的时候的一个小差异\footnote{diff 的输出有多种格式,最广泛接受,人类可读性最好的就是这里使用的统一输出格式。使用 -u 参数获得。}:

{\tiny 
\begin{verbatim}
diff -u a/git.tex b/git.tex
--- a/git.tex
+++ b/git.tex
@@ -120,6 +120,20 @@ Linus决定把他的新版本控制软件设计为一个文件系统,并在此

\end{itemize}

+为了达到他设定的速度目标,Linus决定放弃使用其他版本控制软件惯用的数据库来存储数据,而是直接存储在文件系统上。
+也就是说,每个对象都是直接保存在文件系统里的,而不是像其他版本控制软件那样将数据保存到数据库里。
+Git将对象的SHA1分成两个部分,第一个部分用做目录,第二个部分用作文件名。
+比如一个sha1是f29dee609933c795306eff1075fcb70f2c511a16的对象,
+ 将会保存在f2目录下,文件名是9dee609933c795306eff1075fcb70f2c511a16。
+结合 Git将所有的数据保存在 .git 下考虑,准确的路径则是
+项目路径/.git/objects/f2/9dee609933c795306eff1075fcb70f2c511a16。
+
+随着提交的进行, objects/ 目录下的对象会越来越多。git引入了周期性的打包机制,将松散的包压缩成一个。
+保存在 项目路径的 .git/objects/pack/ 目录下。
+压缩包扩展名 .pack,然后引入一个索引,和压缩包同名,扩展名为 idx。
+索引文件列出了压缩包里包含的对象的sha1。
+这样就能快速的查找对象在哪个包。
+
\subsection{去中心化}
\subsection{GIT典型工作流}
\subsection{GIT 简单使用}
-- 
\end{verbatim}
}

\begin{insertnote}

我简单的讲解一下 diff 输出。

{\scriptsize 
\begin{verbatim}
--- a/git.tex
+++ b/git.tex
\end{verbatim}}

的意思是下面的输出是 git.tex 这个文件的差异。 两个\verb @ \verb @ 开始的一行定义了一个改变的区域。这个区块一直到下一个\verb @ \verb @ 出现为止。\verb @ \verb @后面的数字表示修改的位置,数字为行号,也就是第120行。行号后面的数字6表示修改前这个区块有6行,自然,那个20就是修改后区块有20行了。因为只有这一处改动,所以 +120这个数字还是120,表示修改后的区块位于源文件的第120行。

\end{insertnote}

diff的输出\footnote{其实根本用不着讲解,典型的自解释型输出。}很好的表达了文件内容的变更。如果变动不是太大,似乎靠肉眼阅读就能顺手把文件修改了。

当然,人都是懒的,既然diff的输出是差异,要是有个程序能自动应用这个差异就再好不过了。

当然,执行这个任务的程序就是patch。

\begin{code}
\# diff -uR modifiedproject  myproject-2012-12-21.bak > difftomyproject.diff

\# cd myproject 

\# patch -p1 < ../difftomyproject.diff
\end{code}

注意一下这里的  -p1 参数,主要是看

 {\scriptsize 
\begin{verbatim}
--- a/git.tex
+++ b/git.tex
\end{verbatim}}

如果这里是 a/ b/ 这样的,应用的时候本地因为没有 a/ b/ 目录,所以使用  -p1 去掉第一层假目录。通常使用版本控制软件生成的补丁会带有 a/ b/这样的虚拟前缀。如果没有带这样的前缀,则不需要 -p1 参数。如本例子中的diff,生成的前缀是 modifiedproject/ myproject-2012-12-21.bak/ , 因为当前目录是 myproject,所以要合并的文件应该不需要目录前缀了,故而需要 -p1 去掉前缀。

patch 将 diff 的输出合并,所以diff文件被形象的称呼为补丁\footnote{多数情况下使用 .patch做为扩展名,而不是如例子所用的.diff。}。
这个时候 myproject 就算和  myproject-2012-12-21.bak 有了差异都能合并成功。
只要你和你朋友没有同时修改同一行,合并就能成功。如果你们同时修改了同一个文件的同一个,会导致这个部分无法应用补丁,patch会把无法应用的部分写入到 .rej 结尾的文件中,并提升补丁合并失败。但是能成功合并的地方还是会打上去的。如果认为只有部分合并成功是一种混乱,但是事先又不知道,可以使用 --dry-run 参数。

\begin{code}
\# patch -p1 -dry-run < ../difftomyproject.diff
\end{code}

这样就能不弄坏工作正本的情况下检查补丁是否能成功打上。使用补丁这种形式一直是多年来开源社区相互之间协作的主要方式。

\subsection{版本控制}

虽然原始的办法管用,

当然版本控制不仅仅是后悔药,它还是历史。
通过细化的修订记录,能比较全面的反应项目的整个流程。它忠实的记录了项目被编辑修改的过程。


\section{中心式版本控制}

可能按照时间戳保存副本是一个好主意,但是在在多人协作的时候,这个方法就不管用了。
中心化的版本控制软件应运而生。

TODO 插图

中心化的版本控制软件使用一个中央存储服务器来保存文件的各个版本。其他人只需要和中心服务器进行同步即可。中心化的一个好处就是管理简单,只要在一台 服务器上配置就能控制好所有人的权限——提交的权限和签出的权限。当然,好处也是坏处。集中化的存储要更容易遇到单点故障问题——也就是一台服务器的故障会影响到所有的人。服务器故障一个小时,那么在这一个小时时间段内,所有人都不能提交和签出以及查看历史。如果服务器数据丢失,则项目的整个历史都会丢失。


\subsection{CVS 后继 SVN}

\section{GIT划时代的分布式版本控制软件}\label{sec:git}
 
\subsection{BitKeeper}

Linux内核黑客Larry McVoy在1998年的时候发了一份邮件\footnote{\url{https://lkml.org/lkml/1998/9/30/122}},对Linux内核日益膨胀的代码表达了维护负担的担忧。它希望能有一种工具能极大的结果内核开发者们手中的活,把黑客们从繁重的补丁管理中解放出来。像Linux那么庞大的软件,根本就没有办法使用CVS这样的版本控制。性能不足是非常重要的原因,同时还有CVS本身的各种弊病,导致内核黑客们宁可使用原始的压缩包和补丁来进行管理。但是随着内核体积的膨胀,原始的管理办法越来越力不从心了。维护者开始疲于应对飞到邮箱里的补丁,一些补丁会被丢失,而下游开发者甚至很难知晓自己的补丁是不是丢失了。他们得等到下一个内核发布,并亲子检查后才能确信补丁被接受了。减小这种时间差的办法是加快新版本的发布速度,而这必然会大大加重维护者,特别是Linus的工作负担。

为寻求有效的版本控制系统做的思考下,Larry McVoy设计出了一个名为BitKeeper的版本控制软件,并于2000年正式对外公布了第一个版本。与此同时他成立了公司来销售这一软件——本身是内核开发者的他,选择了开发销售私有软件,而不开源的BitKeeper则是催生Git的重要原因。虽然BitKeeper是一个私有软件,好在Larry McVoy慷慨的为内核开发者提供了“特供”版本。于是从2003年开始BitKeeper正式称为Linux内核的使用版本控制平台。

虽然Larry MocVoy为内核开发提供了特供的免费版本,可是还是有人试图对Bitkeeper进行逆向工程——这直接惹恼了他,
并宣布停止为Linux内核开发者提供免费版本,甚至对参与逆向的人和他所在的公司禁售——连商业版本都不能使用。

虽然RMS大神早早的就说把Linux内核的管理建立这一个专用软件上是非常危险的事情。但是Linus并没有找到合适的替代品,Bitkeeper能提供足够的特性,所以他也用下来了。但是Larry McVoy的举措使得Linus感觉自己似乎应该”寻找个更合适的替代品\footnote{\url{https://lkml.org/lkml/2005/4/6/121} \url{https://lkml.org/lkml/2005/4/7/145}}”了。

\subsection{Linus 消失的一周}

Linus打定主意替换掉Bitkeeper,寻找个更合适的替代品,有必要的话他甚至决定开发一个新的版本控制软件。他对(能做到管理内核)版本控制软件提出了几个苛刻的要求:

\begin{itemize}
\item 不能是CVS\footnote{呐尼?!你没看错,Linus原话是说拿CVS做反例教材。}。Linus认为CVS一切都做错了。凡CVS的方式的都不是他要的版本控制软件。所以CVS的后继者Subversion也不能用。
%\item 开源。这是使用了多年BitKeeper后得出的教训。内核开发人员必须能访问版本控制软件的代码,必须不被“用户许可协议”束缚。
\item 支持分布式工作流,就像BitKeeper一样。分布式意味着每个人都有一份仓库,可以在自己的仓库上做开发。需要的时候把自己的修改同步给其他人。

\item 抵抗数据错误。版本控制软件必须能保护数据正确性。不论是故意的,无意的,还是(被恶意软件)恶意破坏的,都不能毁坏代码和历史。或者破坏了也要能立即检测出。像Linux内核这样的项目,如果其他人获得的代码中途被篡改,用户获得的实际上不是Linus发布的文件,就糟糕了。

\item 速度。内核包含数万个文件,必须能迅速的完成版本控制任务,不能因为历史和代码体积的膨胀造成性能的巨大下滑。

\end{itemize}

前三点所有的开源版本控制软件都被排除了。当时就只剩下Monotone可用了。但是最后一条“速度”则把Monotone也排除了。
既然现有的软件都不能满足自己的需求,秉承著“实在没有就自己造一个”的黑客精神,Linus决定暂时消失一周,离开内核开发前线,去设计一个他满意的版本控制软件。

一周后,Linus重新出现在邮件列表的时候,他带来了一个叫Git的文件系统。

\subsection{版本控制设计为一个文件系统}

Linus决定把他的新版本控制软件设计为一个文件系统,并在此文件系统的基础上提供一个版本控制。Linus把他的新文件系统命名为Git,傻瓜文件内容追踪系统。
在这个傻瓜文件系统里,所有的文件都由一个SHA1哈希值引用。文件在保存的时候就获得一个依据内容生成的SHA1哈希值。而文件的内容则可以使用SHA1哈希值引用。

有了这么一个文件系统,Linus继续设计版本控制:

\begin{itemize}
\item git是一个SHA1对象存储文件系统。每个对象由一个ID标记。ID直接取对象本身的SHA1哈希值。对象分为普通文件对象,树对象,和提交对象。
\item 被版本系统追踪的文件,将其保存在git文件系统里,随后使用其SHA1引用。如果用户编辑了文件,文件的SHA1哈希值随即发生变化,git得以知道用户编辑了该文件。
\item 文件修改,重新保存文件到git文件系统里,而不是将其和上一个版本进行比较,然后保存差异部分。这样的好处就是非常快,重新获取一个文件的内容不需要遍历它的整个历史。根据 其SHA1哈希,可以直接从git文件系统里重新获得文件的内容。
\item git文件系统只追踪文件内容,文件名和文件内容(SHA1)之间的对应关系由树对象描述。
\item 树对象格式是一个文本文件,包含了文件和其他树对象的SHA1哈希值以及文件或目录名。树对象自身的名字则由父级树对象描述。因为该树对象本身也存储于git文件系统里,父级树对象即可使用其SHA1引用。

\item 项目的根目录的SHA1数值代表了项目的一个快照版本。
\item 提交记录用提交对象存储。每个提交对象本身的sha1哈希值,称为commit id。 一个提交记录,就是一个提交信息和一个快照,以及一个或多个父提交的commit id。通常的提交有一个父提交。第一个提交有个特殊的父提交,其值为全0。而有2个父提交的称为合并。将两个版本合并起来,则会有2个父提交。
合并不仅仅可以在两个版本上,多个版本也可以合并。形成八爪鱼效果。
\item 每次提交版本的时候,直接生成新的快照,而不是保存和上一版本的差异。新的快照没有想象中那么占用空间哦~因为如果只改变了一个文件,通常只要快照本身的SHA1和那个改变了的文件的SHA1变化 ,其他文件因为没有修改则继续使用老的SHA1,也就是一个快照事实上并没有占用太多的存储空间。

\end{itemize}

为了达到他设定的速度目标,Linus决定放弃使用其他版本控制软件惯用的数据库来存储数据,而是直接存储在文件系统上。
也就是说,每个对象都是直接保存在文件系统里的,而不是像其他版本控制软件那样将数据保存到数据库里。
Git将对象的SHA1分成两个部分,第一个部分用做目录,第二个部分用作文件名。
比如一个sha1是f29dee609933c795306eff1075fcb70f2c511a16的对象,将会保存在f2目录下,文件名是9dee609933c795306eff1075fcb70f2c511a16。
结合 Git将所有的数据保存在 .git 下考虑,准确的路径则是 项目路径/.git/objects/f2/9dee609933c795306eff1075fcb70f2c511a16。

随着提交的进行, objects/ 目录下的对象会越来越多。git引入了周期性的打包机制,将松散的包压缩成一个。
保存在 项目路径的 .git/objects/pack/ 目录下。
压缩包扩展名 .pack,然后引入一个索引,和压缩包同名,扩展名为 idx。
索引文件列出了压缩包里包含的对象的sha1。
这样就能快速的查找对象在哪个包。

\subsection{去中心化}

Git最突出的特色就是支持分布式的工作流。其他的一些版本控制软件,都需要一个服务器作为中心仓库。其他人都同中心仓库同步。

\subsection{GIT典型工作流}
\subsection{GIT 简单使用}

\chatu{kernelgithistory}{Linux内核历史}


