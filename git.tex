
用电脑编辑文件,当需要保留多个版本的时候,很多人选择拷贝多个备份。这是一种非常原始的版本控制。一个人的时候,这种原始的版本控制顶多会让你决定很麻烦,很混乱。
但是如果两个人以上的团队共同编辑一个项目,版本控制软件就是必须的了。多数版本控制是用来管理软件开发的,但是版本控制用于普通的文件编辑也可以的,只不过在程序员的圈子外很少有人用版本控制。
\footnote{可惜的是他们享受不到版本控制的好处。}

\section{历史和后悔药}

从手指接触键盘伊始,手中被编辑的文件就有了版本。随着时间推移,很多东西加了进来,也有很多东西被Del请了出去。
有一天,突然想起来一个月前删除的那段代码/文字/数据... 需要恢复。

怎么办?一般的编辑器有撤销按钮,可是这可是很久之前的了。再强大的编辑器也无能为力了。

这个时候的版本控制就是你的后悔药。当然版本控制不仅仅是后悔药,通过细化的修改历史,能比较全面的反应项目的整个流程。

\chatu{kernelgithistory}{Linux内核历史}

像内核那么庞大的项目, 很难想像如果没有版本控制会是什么情况。有了版本控制,文件的每一行修改都可以追溯。如果发生bug,也少了嘴皮之争,版本控制明明白白的写清楚了,谁在什么时候修改了哪个文件,修改了哪些行。

\section{   中心式版本控制仓库 CVS	}
\section{   CVS 后继 SVN	}
\section{GIT 划时代的分布式版本控制\label{sec:git}}
\subsection{  Bitkeeper	}
\subsection{  Linus 消失的一周	}
\subsection{  版本控制设计为一个文件系统	}
\subsection{ 去中心化	}
\subsection{  GIT典型工作流	}
\subsection{  GIT 简单使用	}

