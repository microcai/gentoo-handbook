
用电脑编辑文件,当需要保留多个版本的时候,很多人选择拷贝多个备份。这是一种非常原始的版本控制。一个人的时候,这种原始的版本控制顶多会让你决定很麻烦,很混乱。
但是如果两个人以上的团队共同编辑一个项目,版本控制软件就是必须的了。多数版本控制是用来管理软件开发的,但是版本控制用于普通的文件编辑也可以的,像是美术设计啊,小说创作啊,等待。你看到的本书就是使用版本控制进行管理的。
%,只不过在程序员的圈子外少有人用版本控制。\footnote{可惜的是他们享受不到版本控制的好处。}

\section{历史和后悔药}

从手指接触键盘伊始,手中被编辑的文件就有了版本。随着时间推移,很多东西加了进来,也有很多东西被Del请了出去。你一定会对你用着的靠谱的编辑器点下那很靠谱的“\includegraphics[width=1em]{pics/document-save.png}”按钮。好的,有一天,突然想起来一个月前删除的那段代码/文字/数据... 需要恢复。

怎么办?一般的编辑器有撤销按钮,可是这可是很久之前的了。再强大的编辑器也无能为力了。
这个时候的版本控制就是你的后悔药。就算你把整个项目的文件都删除,版本控制都能轻轻松松的帮你找回所有的文件。

当然版本控制不仅仅是后悔药,它还是历史。
通过细化的修订记录,能比较全面的反应项目的整个流程。它忠实的记录了项目被编辑修改的过程。

%像内核那么庞大的项目, 很难想像如果没有版本控制会是什么情况。有了版本控制,文件的每一行修改都可以追溯。如果发生bug,也少了嘴皮之争,版本控制明明白白的写清楚了,谁在什么时候修改了哪个文件,修改了哪些行。

\section{中心式版本控制仓库 CVS}

可能按照时间戳保存副本是一个好主意,但是在在多人协作的时候,这个方法就不管用了。
中心化的版本控制软件应运而生。

TODO 插图

中心化的版本控制软件使用一个中央存储服务器来保存文件的各个版本。其他人只需要和中心服务器进行同步即可。中心化的一个好处就是管理简单,只要在一台 服务器上配置就能控制好所有人的权限——提交的权限和签出的权限。


\section{CVS 后继 SVN}
\section{GIT 划时代的分布式版本控制\label{sec:git}}
\subsection{Bitkeeper}
\subsection{Linus 消失的一周}
\subsection{版本控制设计为一个文件系统}
\subsection{去中心化}
\subsection{GIT典型工作流}
\subsection{GIT 简单使用}

\chatu{kernelgithistory}{Linux内核历史}


