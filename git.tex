
用电脑编辑文件,当需要保留多个版本的时候,很多人选择拷贝多个备份。这是一种非常原始的版本控制。一个人的时候,这种原始的版本控制顶多会让你决定很麻烦,很混乱。
但是如果两个人以上的团队共同编辑一个项目,版本控制软件就是必须的了。多数版本控制是用来管理软件开发的,但是版本控制用于普通的文件编辑也可以的,像是美术设计啊,小说创作啊,等待。你看到的本书就是使用版本控制进行管理的。
%,只不过在程序员的圈子外少有人用版本控制。\footnote{可惜的是他们享受不到版本控制的好处。}

\section{历史和后悔药}

从手指接触键盘伊始,手中被编辑的文件就有了版本。随着时间推移,很多东西加了进来,也有很多东西被Del请了出去。你一定会对你用着的靠谱的编辑器点下那很靠谱的“\includegraphics[width=1em]{pics/document-save.png}”按钮。好的,有一天,突然想起来一个月前删除的那段代码/文字/数据... 需要恢复。

怎么办?一般的编辑器有撤销按钮,可是这可是很久之前的了。再强大的编辑器也无能为力了。
这个时候的版本控制就是你的后悔药。就算你把整个项目的文件都删除,版本控制都能轻轻松松的帮你找回所有的文件。

当然版本控制不仅仅是后悔药,它还是历史。
通过细化的修订记录,能比较全面的反应项目的整个流程。它忠实的记录了项目被编辑修改的过程。

%像内核那么庞大的项目, 很难想像如果没有版本控制会是什么情况。有了版本控制,文件的每一行修改都可以追溯。如果发生bug,也少了嘴皮之争,版本控制明明白白的写清楚了,谁在什么时候修改了哪个文件,修改了哪些行。

\section{中心式版本控制仓库 CVS}

可能按照时间戳保存副本是一个好主意,但是在在多人协作的时候,这个方法就不管用了。
中心化的版本控制软件应运而生。

TODO 插图

中心化的版本控制软件使用一个中央存储服务器来保存文件的各个版本。其他人只需要和中心服务器进行同步即可。中心化的一个好处就是管理简单,只要在一台 服务器上配置就能控制好所有人的权限——提交的权限和签出的权限。当然,好处也是坏处。集中化的存储要更容易遇到单点故障问题——也就是一台服务器的故障会影响到所有的人。服务器故障一个小时,那么在这一个小时时间段内,所有人都不能提交和签出以及查看历史。如果服务器数据丢失,则项目的整个历史都会丢失。


\section{CVS 后继 SVN}
\section{GIT 划时代的分布式版本控制\label{sec:git}}
\subsection{Bitkeeper}

Linux内核黑客Larry McVoy在1998年的时候发了一份邮件\footnote{\url{https://lkml.org/lkml/1998/9/30/122}},对Linux内核日益膨胀的代码表达了维护负担的担忧。它希望能有一种工具能极大的结果内核开发者们手中的活,把黑客们从繁重的补丁管理中解放出来。像Linux那么庞大的软件,根本就没有办法使用CVS这样的版本控制。性能不足是非常重要的原因,同时还有CVS本身的各种弊病,导致内核黑客们宁可使用原始的压缩包和补丁来进行管理。但是随着内核体积的膨胀,原始的管理办法越来越力不从心了。维护者开始疲于应对飞到邮箱里的补丁,一些补丁会被丢失,而下游开发者甚至很难知晓自己的补丁是不是丢失了。他们得等到下一个内核发布,并亲子检查后才能确信补丁被接受了。减小这种时间差的办法是加快新版本的发布速度,而这必然会大大加重维护者,特别是Linus的工作负担。

内核的开发模式的思考使Larry McVoy设计出了一个名为Bitkeeper的版本控制软件,并于2000年正式对外公布了第一个版本。与此同时他成立了公司来销售这一软件——本身是内核开发者的他,选择了开发销售私有软件,而不开源的Bitkeeper则是催生Git的重要原因。虽然Bitkeeper是一个私有软件,好在Larry McVoy慷慨的为内核开发者提供了“特供”版本。于是从2003年开始Bitkeeper正式称为Linux内核的使用版本控制平台。

虽然Larry MocVoy为内核开发提供了特供的免费版本,可是还是有人试图对Bitkeeper进行逆向工程——这直接惹恼了他,
并宣布停止为Linux内核开发者提供免费版本,甚至对参与逆向的人和他所在的公司禁售——连商业版本都不能使用。

虽然RMS大神早早的就说把Linux内核的管理建立这一个专用软件上是非常危险的事情。但是Linus并没有找到合适的替代品,Bitkeeper能提供足够的特性,所以他也用下来了。但是Larry McVoy的举措使得Linus感觉自己似乎应该”寻找个更合适的替代品\footnote{\url{https://lkml.org/lkml/2005/4/6/121} \url{https://lkml.org/lkml/2005/4/7/145}}”了。

\subsection{Linus 消失的一周}

Linus打定主意替换掉Bitkeeper,寻找个更合适的替代品,有必要的话他甚至决定开发一个新的版本控制软件。他对(能做到管理内核)版本控制软件提出了几个苛刻的要求:

\begin{itemize}
\item 快

\end{itemize}

\subsection{版本控制设计为一个文件系统}
\subsection{去中心化}
\subsection{GIT典型工作流}
\subsection{GIT 简单使用}

\chatu{kernelgithistory}{Linux内核历史}


